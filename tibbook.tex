\input tibex

% 缺省的纸张大小为16开
\pdfpageheight26cm\pdfpagewidth18.4cm
\vsize=20.1cm\hsize=13.7cm
% 对于藏文字体行距需要相应的加大
\baselineskip=27pt plus 3pt
%
\font\SamSubt='SambhotaDege:script=tibt;+abvs;+blws' at 21pt
% 定义页眉和页脚
\nopagenumbers
\voffset=1.0cm
\maxdepth=0pt

\newif\iftitle
\def\titlepage{\global\titletrue} % for pages without headlines

\def\plainoutput{\shipout
  \vbox{
     \offinterlineskip % butt the boxes together
     \iftitle
        \global\titlefalse % reset the titlepage switch
     \else
        \makeheadline\headrule
     \fi
     \pagebody \vskip1.1cm 
     \makefootline}%
  \advancepageno  
  \ifnum\outputpenalty>-20000 \else\dosupereject\fi}
%
\def\makeheadline{\vbox to 0pt{
  \vskip-54pt
  \line{\vbox to 20pt{}\the\headline}
  \vss}\nointerlineskip}
%
\def\headrule{\vbox to 0pt{
  \vskip-17pt
  \line{\leaders\hrule height1.5pt\hfill}
  \nointerlineskip
  \vskip2pt
  \line{\leaders\hrule height.7pt\hfill}
  \vss}
}
\def\makefootline{\baselineskip30pt\lineskiplimit0pt\line{\the\footline}}
\def\normalpageno{\footline={\hss\tenrm\folio\hss}}
%
\def\title#1{\centerline{\SamChe #1}}
\def\subtitle#1{\medskip\centerline{\SamSubt #1}\medskip}
% Plain TeX中定义的bigskip对藏文来说过小了,因此我们定义一个更大的
\def\xbigskip{\bigskip\bigskip}
\newdimen\poemskip
\def\setpoemskip#1{\poemskip=#1}
\def\poem#1{{\leftskip=\poemskip #1\par}}
%
% 用来排版超长段的命令
\def\firstpar{\parfillskip=0pt}
\def\followpar{\noindent\parfillskip=0pt}
\def\lastpar{\noindent\parfillskip=0pt plus1fil}
